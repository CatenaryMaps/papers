\documentclass[a4paper,10pt]{article}
\usepackage[utf8]{inputenc}
\usepackage{color}
\usepackage{listings}
\definecolor{GrayCodeBlock}{RGB}{241,241,241}
\definecolor{BlackText}{RGB}{110,107,94}
\definecolor{AquaTypename}{RGB}{45, 191, 184}
\definecolor{BrownString}{RGB}{195,105,57}
\definecolor{AquaKeyword}{RGB}{45, 191, 184}
\definecolor{BlueKeyword}{RGB}{54, 124, 208}
\definecolor{PurpleKeyword}{RGB}{184,84,212}
\definecolor{GrayComment}{RGB}{170,170,170}
\definecolor{GoldDocumentation}{RGB}{180,165,45}
\lstdefinelanguage{rust}
{
    columns=fullflexible,
    keepspaces=true,
    frame=single,
    framesep=0pt,
    framerule=0pt,
    framexleftmargin=4pt,
    framexrightmargin=4pt,
    framextopmargin=5pt,
    framexbottommargin=3pt,
    xleftmargin=4pt,
    xrightmargin=4pt,
    backgroundcolor=\color{GrayCodeBlock},
    basicstyle=\ttfamily\color{BlackText},
    keywords={
        true,false,
        unsafe,async,await,move,
        use,super,self,mod,
        enum,fn,const,static,let,mut,ref,type,impl,dyn,trait,where,as,
        break,continue,if,else,while,for,loop,match,return,yield,in
    },
    keywordstyle=\color{AquaKeyword},
    keywordstyle=[2]{
        pub,struct,extern,crate
    },
    keywordstyle=[2]\color{BlueKeyword},
    ndkeywords={
        bool,u8,u16,u32,u64,u128,i8,i16,i32,i64,i128,char,str,
        Self,Option,Some,None,Result,Ok,Err,String,Box,Vec,Rc,Arc,Cell,RefCell,HashMap,BTreeMap,
        FeedMessage,UserFeed,
        macro_rules
    },
    ndkeywordstyle=\color{AquaTypename},
    comment=[l][\color{GrayComment}\slshape]{//},
    morecomment=[s][\color{GrayComment}\slshape]{/*}{*/},
    morecomment=[l][\color{GoldDocumentation}\slshape]{///},
    morecomment=[s][\color{GoldDocumentation}\slshape]{/*!}{*/},
    morecomment=[l][\color{GoldDocumentation}\slshape]{//!},
    morecomment=[s][\color{AquaTypename}]{\#![}{]},
    morecomment=[s][\color{AquaTypename}]{\#[}{]},
    stringstyle=\color{BrownString},
    string=[b]"
}

\title{Algorithms for Public Transit Graphs}
\author{Joshua Wong, Kyler Chin, Jaiveer Gahunia}
\author{
  \small Joshua Wong \\
  \small Catenary Maps \\
  \small \texttt{admin@catenarymaps.org}
  \and
  \small Kyler Chin \\
  \small Catenary Maps \\
  \small \texttt{kyler@catenarymaps.org}
  \and Jaiveer Gahunia
  \and Chelsea Wen
}

\begin{document}

\maketitle

\begin{abstract}

\end{abstract}

\section{Introduction}
These algorithms will be used to construct a public transit system as a graph.
We will use the WGS84 coordinate system and represent the Earth as an oblate spheroid. The difficulty in these problems comes from the issue of the Earth's irregular shape.
\section{Point-Geodesic Problem}
The problem states: Given a Geodesic, defined by the shortest path between two points on a sphere's surface, and a random point on the sphere's surface, find the closest point on that line to that random point.
\subsection{Solution in Euclidean Space}
Take Points $A$ and $B$, and parametrize $\overline{AB}$ to create a function of a vector, $f(t)$. Derive $f(t)$ to find the critical point, $t_c$. The closest point will be $f(t_c)$.
\subsection{Kyler's Solution}
Take the Geodesic's endpoints $(P_1, P_2)$, along with the random point, $R_1$, and construct $\triangle{P_1P_2R_1}$, using Vincenty's formulae to construct the edges. Then, take the midpoint of $\overline{P_1P_2}$, and construct a new line using this new midpoint, $M_1$, and $R_1$. Throw out whichever line is longer $\overline{R_1P_1}$ or $\overline{R_1P_2}$, and replace it with $\overline{R_1M_1}$. Repeat this process until $\overline{P_1P_2}$ is within the specified tolerance. This algorithm is $O(\log_2(n))$, where $n$ is equal to the original length of $\overline{P_1P_2}$. It takes $\log_2(n)$ steps to reach the specified tolerance.
\subsection{Final Solution}
We just ended up using Baselga and Martínez-Llario's Solution on this problem \cite{Baselga2018}, with Karney's improvements. \cite{karney2023geodesic}
\bibliographystyle{IEEEtran}
\bibliography{algo.bib}


\end{document}
