\documentclass[a4paper,10pt]{article}
\usepackage[utf8]{inputenc}

%opening
\title{Nearest Point-Geodesic Problem}
\author{Joshua Wong, Kyler Chin}

\begin{document}

\maketitle

\begin{abstract}
The problem states: Given a Geodesic, defined by the shortest path between two points on a sphere's surface, and a random point on the sphere's surface, find the closest point on that line to that random point.
\end{abstract}
\section{Introduction}
We will Use the wgs84 coordinate system, and repersent the Earth as a oblate spheroid. The difficulty in this problem comes from the issue of the Earth's irregular shape.
\section{Kyler's Solution}
Take the Geodesic's endpoints $(P_1, P_2)$, along with the random point, $R_1$, and construct a triangle, $\triangle{P_1P_2R_1}$, using Vincenty's formulae to construct the edges. Then, take the midpoint of $\overline{P_1P_2}$, and construct a new line using this new midpoint, $M_1$ and $R_1$. Throw out whichever line is longer $\overline{R_1P_1}$ or $\overline{R_1P_2}$, and replace it with $\overline{R_1M_1}$. Repeat this process until $\overline{P_1P_2}$ is within the specified tolerance. This algorithm is $O(\log_2(n))$, where $n$ is equal to the orignal length of $\overline{P_1P_2}$. It takes $log_2(n)$ steps to reach the specified tolerance.
\end{document}
