\documentclass[a4paper,10pt]{article}
\usepackage[utf8]{inputenc}
\usepackage{color}
\usepackage{listings}
\definecolor{GrayCodeBlock}{RGB}{241,241,241}
\definecolor{BlackText}{RGB}{110,107,94}
\definecolor{RedTypename}{RGB}{182,86,17}
\definecolor{GreenString}{RGB}{96,172,57}
\definecolor{PurpleKeyword}{RGB}{184,84,212}
\definecolor{GrayComment}{RGB}{170,170,170}
\definecolor{GoldDocumentation}{RGB}{180,165,45}
\lstdefinelanguage{rust}
{
    columns=fullflexible,
    keepspaces=true,
    frame=single,
    framesep=0pt,
    framerule=0pt,
    framexleftmargin=4pt,
    framexrightmargin=4pt,
    framextopmargin=5pt,
    framexbottommargin=3pt,
    xleftmargin=4pt,
    xrightmargin=4pt,
    backgroundcolor=\color{GrayCodeBlock},
    basicstyle=\ttfamily\color{BlackText},
    keywords={
        true,false,
        unsafe,async,await,move,
        use,pub,crate,super,self,mod,
        struct,enum,fn,const,static,let,mut,ref,type,impl,dyn,trait,where,as,
        break,continue,if,else,while,for,loop,match,return,yield,in
    },
    keywordstyle=\color{PurpleKeyword},
    ndkeywords={
        bool,u8,u16,u32,u64,u128,i8,i16,i32,i64,i128,char,str,
        Self,Option,Some,None,Result,Ok,Err,String,Box,Vec,Rc,Arc,Cell,RefCell,HashMap,BTreeMap,
        macro_rules
    },
    ndkeywordstyle=\color{RedTypename},
    comment=[l][\color{GrayComment}\slshape]{//},
    morecomment=[s][\color{GrayComment}\slshape]{/*}{*/},
    morecomment=[l][\color{GoldDocumentation}\slshape]{///},
    morecomment=[s][\color{GoldDocumentation}\slshape]{/*!}{*/},
    morecomment=[l][\color{GoldDocumentation}\slshape]{//!},
    morecomment=[s][\color{RedTypename}]{\#![}{]},
    morecomment=[s][\color{RedTypename}]{\#[}{]},
    stringstyle=\color{GreenString},
    string=[b]"
}

\title{Nearest Point-Geodesic Problem}
\author{Joshua Wong, Kyler Chin}

\begin{document}

\maketitle

\begin{abstract}
The problem states: Given a Geodesic, defined by the shortest path between two points on a sphere's surface, and a random point on the sphere's surface, find the closest point on that line to that random point.
\end{abstract}

\section{Introduction}
We will use the WGS84 coordinate system and represent the Earth as an oblate spheroid. The difficulty in this problem comes from the issue of the Earth's irregular shape.

\section{Solution in Euclidean Space}
\section{Kyler's Solution}
Take the Geodesic's endpoints $(P_1, P_2)$, along with the random point, $R_1$, and construct $\triangle{P_1P_2R_1}$, using Vincenty's formulae to construct the edges. Then, take the midpoint of $\overline{P_1P_2}$, and construct a new line using this new midpoint, $M_1$, and $R_1$. Throw out whichever line is longer $\overline{R_1P_1}$ or $\overline{R_1P_2}$, and replace it with $\overline{R_1M_1}$. Repeat this process until $\overline{P_1P_2}$ is within the specified tolerance. This algorithm is $O(\log_2(n))$, where $n$ is equal to the original length of $\overline{P_1P_2}$. It takes $\log_2(n)$ steps to reach the specified tolerance.

\begin{lstlisting}[language=rust]
fn main() {
    println!("Hello, world!");
}
\end{lstlisting}

\end{document}
